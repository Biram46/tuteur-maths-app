\documentclass{article}
\usepackage[utf8]{inputenc}
\usepackage[french]{babel}
\usepackage{amsmath, amssymb}

\title{Cours : Le Second Degré}
\author{Tuteur Maths}
\date{Janvier 2026}

\begin{document}

\maketitle

\section{Fonction Polynôme du Second Degré}
Une fonction polynôme du second degré est définie sur $\mathbb{R}$ par :
\[ f(x) = ax^2 + bx + c \]
avec $a \neq 0$.

\subsection{Forme canonique}
\[ f(x) = a(x - \alpha)^2 + \beta \]
où $\alpha = -\frac{b}{2a}$ et $\beta = f(\alpha)$.

\section{Résolution d'équations}
On calcule le discriminant : $\Delta = b^2 - 4ac$.
\begin{itemize}
    \item Si $\Delta > 0$ : deux racines $x_1, x_2 = \frac{-b \pm \sqrt{\Delta}}{2a}$.
    \item Si $\Delta = 0$ : une racine double $x_0 = -\frac{b}{2a}$.
    \item Si $\Delta < 0$ : aucune racine réelle.
\end{itemize}

\end{document}
